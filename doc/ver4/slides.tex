\documentclass{beamer}

\usepackage{xcolor,colortbl}
\usepackage{polyglossia}
\usepackage{fontspec}
\usepackage{nameref}
\usepackage{ifthen}

\usepackage{biblatex}

\usefonttheme{professionalfonts}
\usetheme{Antibes}
\useoutertheme{infolines_foot}
\setbeamercovered{transparent=20}

\setbeamertemplate{blocks}[default]
\setbeamercolor{block title}{use={frametitle},fg=frametitle.fg,bg=frametitle.bg}

\usepackage[math-style=ISO,vargreek-shape=unicode]{unicode-math}
\setdefaultlanguage{english}

\defaultfontfeatures{Ligatures={TeX}}
\setmainfont{CMU Serif}
\setsansfont{CMU Sans Serif}
\setmonofont{CMU Typewriter Text}
\setmathfont{Latin Modern Math}
\AtBeginDocument{\renewcommand{\setminus}{\mathbin{\backslash}}}

\makeatletter
\newcommand*{\currentname}{\@currentlabelname}
\makeatother
\def\t{\texttt}

\newcommand{\cimg}[2]{%
	\begin{center}%
		\ifthenelse{\equal{#2}{}}{%
			\includegraphics[width=0.75\linewidth]{#1}
		}{%
			\includegraphics[width=#2\linewidth]{#1}
		}%
	\end{center}%
}

\AtNextBibliography{\tiny}


\title[RNA-seq scaffolding]{Scaffold graph visualization and RNA-Seq scaffolding}
\author[Chernikova Olga]{Chernikova Olga\\
	Supervisor: Andrey Prjibelski}
\institute[CAB]{Center for Algorithmic Biotechnology}
\date{07/27/2017}

\begin{document}

\begin{frame}
	\titlepage
\end{frame}

\section{Introduction}

\begin{frame}[t]{Genome assembly}
\cimg{p1.png}{0.68}
\end{frame}

\begin{frame}[t]{Genome assembly}
\cimg{p2.png}{0.68}
\end{frame}

\begin{frame}[t]{Connection using DNA paired-end reads}
	\begin{itemize}
		\item Paired reads:  
		\cimg{pairRead.png}{1}
		\item Finding connections using paired reads: 
		\cimg{pairReadAlig}{1}
	\end{itemize}
\end{frame}

\begin{frame}[t]{Using RNA-seq reads}
\begin{center}
\cimg{rna.png}{1.24}
\end{center}
\end{frame}

\begin{frame}[t]{Using RNA-seq reads}
\begin{center}
\cimg{rnaReads.png}{1.24}
\end{center}
\end{frame}

\section{Goal and tasks}
\begin{frame}[t]{Goal and tasks}
	\begin{block}{Goal}
	Build scaffolds using RNA-seq reads 
	\end{block}
	\begin{block}{Tasks}
	\begin{itemize}
		\item Build the scaffold graph
		\item Build scaffolds using obtained connections
		\item Create tool for visualizing a scaffold graph
		\item Compare results with other tools
	\end{itemize}
\end{block}
\end{frame}	

\section{Existing RNA-seq scaffolders}
\begin{frame}[t]{Existing RNA-seq scaffolders}
\begin{itemize}
	\item {\bf L\_RNA\_scaffolder} (2013) uses long transcriptome reads.
	\item {\bf AGOUTI} (2015) uses paired-end RNA-seq reads and gene predictions.
	\item {\bf rascaf} (2016) uses paired-end RNA-seq reads.
	\item {\bf P\_RNA\_scaffolder} uses paired-end RNA-seq reads, unpublished, details unknown. 
\end{itemize}
\end{frame}

\section{Tasks solving}
\subsection{Visualization}

\begin{frame}[t]{Visualization}
%связь между контигами, dot формат, разбиваем граф на несколько файликов. 
\cimg{examp.jpg}{0.8}
\end{frame}

\begin{frame}[t]{Visualization}
%Какие виды связи можно визуализировать. Референс, РНК, ДНК, скаффолды. 
\cimg{diflib.png}{1.03}
\end{frame}


\begin{frame}[t]{Visualization}
Drawing interesting parts:
\begin{itemize}
	\item difference between two libs 
	\item one lib is present and another is not
	\item difference with reference (possible missassembly)
	\item etc
\end{itemize}
	%Варианты фильтрация +  несколько слов, для чего можно использовать 
\end{frame}

\subsection{Scaffolding}

\begin{frame}[t]{Building a scaffold graph}
	\begin{itemize}
		\item Align RNA-seq paired-end reads
		\item Build a scaffold graph using these alignments
		\item Split every read into two parts in the middle 
		\item Align these parts independently
		\item Build graph using these parts of the reads
		\item Save graph
	\end{itemize}	
	\cimg{rnaReadsCon.png}{1}
\end{frame}

\begin{frame}[t]{Graph simplification}
	\begin{itemize}
		\item Delete low-weight edges 
		\cimg{wrong.png}{0.8}
		\only<2->{\item Edge projection}
		\only<2->{\cimg{outinline.png}{1}}
	\end{itemize}
\end{frame}

\begin{frame}[t]{Graph simplification}
	\begin{itemize}
		\item Delete cycles
		\cimg{delCycl.png}{0.35}
		\only<2->{\item Fork with big difference in weight}
		\only<2->{\cimg{bigDif.png}{0.35}}
		\only<3->{\item Connect simple paths into scaffolds}
		\only<3->{\cimg{gline.png}{1}}
	\end{itemize}
\end{frame}

\begin{frame}[t]{Сomparison} 
	C.elegans, SRR1560107
	\begin{center}
		\begin{tabular}{|l|>{\columncolor[gray]{0.8}}c|c|c|c|}
			\hline
			&bio\_scaffolder&P\_RNA\_scaffolder&rascaf&AGOUTI\\
			\hline
			NG50&36855&36075&32879&31714\\
			\hline
			NG75&17299&17188&18395&14452\\
			\hline
			NGA50&30383&28828&27116&25384\\
			\hline
			NGA75&12735&12489&11667&10246\\
			\hline
			LGA50&918&955&995&1025\\
			\hline
			misassemblies&529&621&521&545\\
			\hline
		\end{tabular}
		\printbibliography
		
	\end{center}
\end{frame}

\section{Results}
\begin{frame}[t]{Results and plans}
	\begin{block}{Results}
		\begin{itemize}
			\item Designed an algorithm for RNA-seq scaffolding 
			\item Developed a tool for visualization a scaffold graph
		\end{itemize}
	\end{block}
	\begin{block}{Plans}
		\begin{itemize}
			\item Compare and test against other tools on more datasets
			\item Improve performance
			\item Implement new ideas for scaffolding  
			\item Write manual and interface
			\item Write a paper
			\end{itemize}
	\end{block}
\end{frame}


\section{Thank you!}
\begin{frame}{Thank you!}
    \begin{center}
        https://github.com/olga24912/bio\_scaffolder
    \end{center}
\end{frame}
\end{document}