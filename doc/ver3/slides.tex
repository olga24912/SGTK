\documentclass{beamer}

\usepackage{xcolor,colortbl}
\usepackage{polyglossia}
\usepackage{fontspec}
\usepackage{nameref}
\usepackage{ifthen}

\usefonttheme{professionalfonts}
\usetheme{Antibes}
\useoutertheme{infolines_foot}
\setbeamercovered{transparent=20}

\setbeamertemplate{blocks}[default]
\setbeamercolor{block title}{use={frametitle},fg=frametitle.fg,bg=frametitle.bg}

\usepackage[math-style=ISO,vargreek-shape=unicode]{unicode-math}
\setdefaultlanguage[spelling=modern,babelshorthands=true]{russian}
\setotherlanguage{english}

\defaultfontfeatures{Ligatures={TeX}}
\setmainfont{CMU Serif}
\setsansfont{CMU Sans Serif}
\setmonofont{CMU Typewriter Text}
\setmathfont{Latin Modern Math}
\AtBeginDocument{\renewcommand{\setminus}{\mathbin{\backslash}}}

\makeatletter
\newcommand*{\currentname}{\@currentlabelname}
\makeatother
\def\t{\texttt}

\newcommand{\cimg}[2]{%
	\begin{center}%
		\ifthenelse{\equal{#2}{}}{%
			\includegraphics[width=0.75\linewidth]{#1}
		}{%
			\includegraphics[width=#2\linewidth]{#1}
		}%
	\end{center}%
}

\title[Скаффолды по ридам РНК]{Построение скаффолдов по ридам РНК и визуализация графа связей между контигами}
\author[Черникова Ольга]{Черникова Ольга\\
	Руководитель: Пржибельский Андрей}
\institute[CAB]{Центр алгоритмических биотехнологий}
\date{27.07.2017}

\begin{document}

\begin{frame}
	\titlepage
\end{frame}

\section{Введение в предметную область}

\begin{frame}[t]{Задача сборки генома}
\cimg{p1.png}{0.68}
\end{frame}

\begin{frame}[t]{Задача сборки генома}
\cimg{p2.png}{0.68}
\end{frame}

\section{Методы нахождения связей}

\begin{frame}[t]{По парным ридам ДНК}
	\begin{itemize}
		\item Парные риды:  
		\cimg{pairRead.png}{1}
		\item Нахождение связей с помощью парных ридов: 
		\cimg{pairReadAlig}{1}
	\end{itemize}
\end{frame}

\begin{frame}[t]{По ридам РНК}
\begin{center}
\cimg{rna.png}{1.24}
\end{center}
\end{frame}

\begin{frame}[t]{По ридам РНК}
\begin{center}
\cimg{rnaReads.png}{1.24}
\end{center}
\end{frame}

\section{Цель и задачи}
\begin{frame}[t]{Цель и задачи}
	\begin{block}{Цель}
	Построение скаффолдов по ридам РНК 
	\end{block}
	\begin{block}{Задачи}
	\begin{itemize}
		\item Построение графа связей
		\item Построение скаффолдов по полученным связям
		\item Создание инструмента для визуализации графа связей между контигами
		\item Сравнение получившихся результатов с результатами других инструментов для построения скаффолодов по ридам РНК  
	\end{itemize}
\end{block}
\end{frame}	

\section{Решение поставленных задач}
\subsection{Визуализация}

\begin{frame}[t]{Визуализация}
%связь между контигами, dot формат, разбиваем граф на несколько файликов. 
\cimg{examp.jpg}{0.8}
\end{frame}

\begin{frame}[t]{Визуализация}
	%Какие виды связи можно визуализировать. Референс, РНК, ДНК, скаффолды. 
	\cimg{diflib.png}{1.03}
\end{frame}


\begin{frame}[t]{Визуализация}
Возможности для фильтрации графа:
\begin{itemize}
	\item по весу ребер и размеров контигов
	\item вывод только участков с разницей в двух библиотеках
	\item только участков, где есть одна библиотека и нет второй 
	\item только участков с ошибочными соединениями
	\item и т.д.
\end{itemize}

	%Варианты фильтрация +  несколько слов, для чего можно использовать 
\end{frame}

\subsection{Построение скаффолдов}

\begin{frame}[t]{Построение графа связей}
	\begin{itemize}
		\item выравнивание парных ридов РНК
		\item построения графа связей по парным ридам
		\item разрезание ридов на две части
		\item выравнивание половин ридов РНК
		\item построение графа по половинам ридов
		\item сохранение графа
	\end{itemize}
	%Использованние связей между контигами.	
	\cimg{rnaReadsCon.png}{1}
\end{frame}

\begin{frame}[t]{Упрощение графа}
	\begin{itemize}
		\item удаление ребер маленького веса
		\cimg{wrong.png}{0.8}
		\only<2->{\item проекция ребер}
		\only<2->{\cimg{outinline.png}{1}}
	\end{itemize}
	
	%\cimg{right.png}{0.8}
	
	%\begin{itemize}
	%	\item<1-> Простые пути
%		\cimg{gline.png}{1}
%		\only<2->{\item<2-> Ребра вдоль пути}
%		\only<2->{}
%		\only<3->{\item<3-> Развилка}
%		\only<3->{\cimg{gsplit.png}{1}}
%	\end{itemize}

\end{frame}

\begin{frame}[t]{Упрощение графа}
	\begin{itemize}
		\item удаление циклов
		\cimg{delCycl.png}{0.4}
		\only<2->{\item развилки с большой разницей в весе}
		\only<2->{\cimg{bigDif.png}{0.4}}
		\only<3->{\item соединение простых путей в скаффолды}
		\only<3->{\cimg{gline.png}{1}}
	\end{itemize}
\end{frame}

\begin{frame}[t]{Cравнение}
C.elegans, SRR1560107
\begin{center}
\begin{tabular}{|l|>{\columncolor[gray]{0.8}}c|c|c|}
	\hline
	&bio\_scaffolder&P\_RNA\_scaffolder&rascaf\\
	\hline
	NG50&36855&36075&32879\\
	\hline
	NG75&17299&17188&18395\\
	\hline
	NGA50&30383&28828&27116\\
	\hline
	NGA75&12735&12489&11667\\
	\hline
	LGA50&918&955&995\\
	\hline
	misassemblies&529&621&521\\
	\hline
\end{tabular}
\end{center}
\end{frame}

\section{Использованные инструменты}
\begin{frame}[t]{Использованные инструменты}
	\begin{itemize}
		\item Язык разработки - \textbf{С++}
		\item \textbf{SeqAn} - библиотека для работы с 
		файлами в SAM/BAM и fasta/fastq форматах.  
		\item \textbf{gtest} - библиотека для тестирования.
		\item Программы для выранивания - \textbf{STAR, nucmer, bowtie2}. 
		\item \textbf{QUAST} - для анализа качетсва сборки. 
		\item \textbf{Tablet} - для визуализации выравненых ридов. 
	\end{itemize}
\end{frame}

\section{Результаты}
\begin{frame}[t]{Результаты и планы}
	\begin{block}{Результаты}
		\begin{itemize}
			\item Создание программы для построения скаффолдов по данным РНК 
			\item Создание инструмента для визуализации графа сязей%, который помогает анализировать
			%корректность построенных соединений   
		\end{itemize}
	\end{block}
	\begin{block}{Дальнейшее развитие}
		\begin{itemize}
			\item Тестирование и сравнение на большем разнообразии данных
			\item Ускорение работы приложения  
			\item Реализация новых идей для построения скаффолдов 
			\item Написание документации и удобного интерфейса
			\item Написание статьи
			\end{itemize}-
	\end{block}
\end{frame}


\section{Спасибо за внимание}
\begin{frame}{Спасибо за внимание}
    \begin{center}
        Репозиторий: https://github.com/olga24912/bio\_scaffolder
    \end{center}
\end{frame}
\end{document}